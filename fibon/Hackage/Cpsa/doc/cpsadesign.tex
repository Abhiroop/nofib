\documentclass[12pt]{report}
\usepackage{amssymb}
\usepackage{amsmath}
\usepackage{amsthm}
\usepackage{url}
\usepackage{graphicx}
\usepackage{makeidx}
\theoremstyle{definition}
\newtheorem{defn}{Definition}[chapter]
\newtheorem{thm}{Theorem}[chapter]
\newtheorem{lem}[thm]{Lemma}
\newcommand{\cpsa}{\textsc{cpsa}}
\newcommand{\version}{2.0.6}
\newcommand{\cpsacopying}{\begingroup
  \renewcommand{\thefootnote}{}\footnotetext{{\copyright} 2010 The
    MITRE Corporation.  Permission to copy without fee all or part of
    this material is granted provided that the copies are not made or
    distributed for direct commercial advantage, this copyright notice
    and the title of the publication and its date appear, and notice
    in given that copying is by permission of The MITRE
    Corporation.}\endgroup}
\newcommand{\pov}{\textsc{pov}}
\newcommand{\cow}{\textsc{cow}}
\newcommand{\cn}[1]{\ensuremath{\operatorname{\mathsf{#1}}}}
\newcommand{\dom}[1]{\ensuremath{\operatorname{\mathrm{#1}}}}
\newcommand{\fn}[1]{\ensuremath{\operatorname{\mathit{#1}}}}
\newcommand{\seq}[1]{\ensuremath{\langle#1\rangle}}
\newcommand{\enc}[2]{\ensuremath{\{\!|#1|\!\}_{#2}}}
\newcommand{\comp}[2]{{\mathord{#1}}\circ{\mathord{#2}}}
\newcommand{\sembrack}[1]{[\![#1]\!]}
\newcommand{\semfn}[1]{\mathcal{#1}}
\newcommand{\sem}[2]{\semfn{#1}\sembrack{#2}}
\newcommand{\mcsu}{\ensuremath{\mathcal{C}}}
\newcommand{\probs}{\ensuremath{\mathcal{E}}}
\newcommand{\eqq}{\stackrel{?}{=}}
\newcommand{\inbnd}{\mathord -}
\newcommand{\outbnd}{\mathord +}
\newcommand{\nat}{\mathbb{N}}
\newcommand{\all}[1]{\forall#1\mathpunct{.}}
\newcommand{\some}[1]{\exists#1\mathpunct{.}}
\newcommand{\baseterms}{\mathbb{B}}
\newcommand{\cons}{\mathbin{::}}
\newcommand{\ith}{\imath^\mathrm{th}}
\newcommand{\jth}{\jmath^\mathrm{th}}
\newcommand{\append}{\mathbin{{}^\smallfrown}}
\newcommand{\prefix}[2]{#1|_{#2}}
\newcommand{\orig}{\mathcal{O}}
\newcommand{\gain}{\mathcal{G}}
%\newcommand{\pow}[1]{\mathcal{P}(#1)}
%\newcommand{\pow}[1]{\wp(#1)}
\newcommand{\pow}[1]{2^{#1}}
\newcommand{\termat}{\mathbin{@}}
\newcommand{\idsigma}{\sigma_{\mathrm{id}}}
\newcommand{\idphi}{\phi_{\mathrm{id}}}
\newcommand{\kprec}[1]{\mathbin{\prec_{#1}}}
\newcommand{\terms}{\ensuremath{\mathcal{T}}}
\newcommand{\alg}[1]{\ensuremath{\mathfrak#1}}
\newcommand{\ops}[1]{\ensuremath{\mathbb#1}}
\newcommand{\bundle}{\ensuremath{\mathcal{B}}}
\newcommand{\nncow}[1]{\mathbin{\odot^{#1}}}
\newcommand{\ncow}[1]{\mathbin{\dagger^{#1}}}
\newcommand{\der}[3]{\ensuremath{\dctx{#2}{#3}\vdash#1}}
\newcommand{\dctx}[2]{\ensuremath{#1:#2}}
\newcommand{\flow}[3]{\ensuremath{#1,#2\rhd#3}}
\newcommand{\infer}[2]{\begin{array}{c}#1\\\hline#2\end{array}}
\hyphenation{pre-skel-e-ton}
\newcommand{\homomorphism}[1]{\stackrel{#1}{\longmapsto}}
\newcommand{\reduction}[1]{\stackrel{#1}{\longrightarrow}}
\newcommand{\longtwoheadrightarrow}{\relax
\smash{\relbar\joinrel\twoheadrightarrow}\vphantom\rightarrow}
\newcommand{\setreduction}[1]{\stackrel{#1}{\longtwoheadrightarrow}}


\makeindex

\title{CPSA Design}
\author{John D.~Ramsdell\qquad Joshua D.~Guttman\\ The MITRE Corporation}

\begin{document}
\maketitle
\cpsacopying

\tableofcontents

\chapter{Introduction}

The Cryptographic Protocol Shapes Analyzer ({\cpsa}) enumerates all
essentially different executions possible for a cryptographic
protocol.  We call them the \emph{shapes} of the protocol.  Naturally
occurring protocols have only finitely many, indeed very few shapes.
Authentication and secrecy properties are easy to determine from them,
as are attacks and anomalies.

The shapes analysis is performed within a pure Dolev-Yao model.  The
{\cpsa} program reads a sequence of problem descriptions, and prints
the steps it used to solve each problem.  For each input problem,
{\cpsa} is given some initial behavior, and it descovers what shapes
are compatible with it.  Normally, the initial behavior is from the
point of view of one participant.  The analysis reveals what the other
participants must have done, given the participant's view.  The search
is complete, i.e.\ every shape can in fact be found in a finite
number of steps.

A {\cpsa} release includes two other documents, The {\cpsa}
Specification and The {\cpsa} Primer.  The specification describes the
{\cpsa} algorithm in a form that is closely related to its
implementation.  Algorithm correctness proofs are given in the
specification formalism.

There are many design decisions that are not reflected in The {\cpsa}
Specification.  Including these decisions in the specification would
clutter the document and unnecessarily complicate proofs.  The purpose
of this document is to describe the key omitted design decisions.  It
assumes the specification has been thoroughly read.  Definitions are
not reproduced, so the specification should be accessible when reading
this document.  The {\cpsa} Primer provides an overview of {\cpsa},
and may be worth reading before this document is approached.

\section{Notation}\label{sec:notation}

Originally, the specification and the design were one and the same.
Everything was specified in the design formalism.  After the split,
the description of protocols and preskeletons diverged by omitting
details in the design formalism in what is used for the specification
formalism.  Appendix~A in The {\cpsa} Specification relates the two
formalisms.

The key difference between the two formalisms is the design formalism
more directly models the Haskell data structures used in the {\cpsa}
program.  An instance of a Haskell data structure is modeled as an
element in an order-sorted term algebra~\cite{GoguenMeseguer92}.  The
reduction systems in the specification translate to term reduction
systems in the design.

Zero-based indexing\index{zero-based indexing}\index{indexing,
  zero-based} is used though out this document and in the source code
it describes.  Within the document, a finite sequence is a function
from an initial segment of the natural numbers.  Angle brackets are
used for sequence construction and~$\append$ is used for sequence
concatenation.  Thus $$\seq{3,2}\append\seq{99} = \{0\mapsto 3,
1\mapsto 2, 2\mapsto 99\}.$$ The length of a sequence~$S$ is $|S|$.
The phrase ``for~$x\in S$'' is shorthand for ``for~$x$ in the range
of~$S$''.  The mapping of sequence~$S$ by function~$f$ is~$f\circ S$.
Some sets are represented by sequences with no duplicates, and~$f(S)$
is used for~$f\circ S$ after duplicates have been eliminated.

You might wonder why lowercase $k$ is used for skeletons rather than
blackboard bold $\mathbb{A}$.  The notation used in the design
document is motivated by the code, and~$k$ with and without decoration
somehow became associated with preskeletons, probably because $p$,
$r$, $t$, and $s$, were already in use.  Many other notational
conventions are directly inspired by the code.

In this document, lowercase Latin letters usually stand for terms, and
uppercase Latin letters stand for sequences or sets of terms.

\section{Document Status}

This version of the design describes the {\cpsa} implementation that
assumes the unification problem for the message algebra has one most
general unifier.  The Basic Crypto Algebra has this property.

\chapter{Messages}\label{cha:messages}

The formalism used in the design and the specification for message
algebras is the same, an order-sorted term algebra.  This chapter
describes the relation between terms and the external syntax used by
the {\cpsa} program for the Basic Crypto Algebra, and then describes
the interface between the algebra module and the rest of the program.

Table~\ref{tab:bca} presents a slightly modified signature for the
Basic Crypto Algebra.  It specifies a syntax for operations that
follows mathematical tradition, such as writing $K_A$ for
$\cn{pubk}(A)$.  Tag constants are quoted strings.  The sort of all
messages is \dom{mesg}, the sort symbol used by the {\cpsa} program.
The sort~$\top$ is used in the specification so that the carrier set
of all messages is~$\alg{A}_\top$, instead of $\alg{A}_{\dom{mesg}}$.

\begin{table}
\begin{center}
Base sort symbols: \dom{name}, \dom{text}, \dom{data}, \dom{skey},
\dom{akey}\\
Non-base sort symbol: \dom{mesg} \\[1ex]
Subsorts: \dom{name}, \dom{text}, \dom{data}, \dom{akey},
$\dom{skey}<\dom{mesg}$\\[1ex]
\begin{tabular}{@{}ll}
$\enc{\cdot}{(\cdot)}\colon\dom{mesg}\times\dom{mesg}\rightarrow\dom{mesg}$
&Encryption\\
$(\cdot,\cdot)\colon\dom{mesg}\times\dom{mesg}\rightarrow\dom{mesg}$
&Pairing\\
``\ldots''$\colon\dom{mesg}$& Tag constants\\
$K_{(\cdot)}\colon\dom{name}\rightarrow\dom{akey}$
&Public key of name\\
$(\cdot)^{-1}\colon\dom{akey}\rightarrow\dom{akey}$
&Inverse of asymmetric key\\
$\cn{ltk}\colon\dom{name}\times\dom{name}\rightarrow\dom{skey}$
& Long term shared key
\end{tabular}\\[1ex]
Axiom: $(x^{-1})^{-1}\approx x$ for $x\colon\dom{akey}$\\[1ex]
\caption{Basic Crypto Signature}\label{tab:bca}
\end{center}
\end{table}

For pairing, parentheses are omitted when the context permits, and
comma is right associative.  Pairing was once called concatenation,
hence the use of the symbol \texttt{cat} for pairing.

In the S-expression syntax used by the program, the simplest term is a
variable, which syntactically is a \textsc{symbol} as described in
Appendix~\ref{cha:bca syntax reference}.  Internally, each variable
has a sort, so the sort of each variable in the input must be declared
in a \texttt{vars} form, such as:
$$\texttt{(vars (t text) (n name) (k akey))}.$$

\begin{table}
$$\begin{array}{r@{}c@{}l}
\sembrack{\texttt{(pubk~}t\texttt{)}}&{}={}&K_{\sembrack{t}}\\
\sembrack{\texttt{(privk~}t\texttt{)}}&{}={}&K_{\sembrack{t}}^{-1}\\
\sembrack{\texttt{(invk~}t\texttt{)}}&{}={}&\sembrack{t}^{-1}\\
\sembrack{\texttt{(ltk~}t_0~t_1\texttt{)}}&{}={}&
\cn{ltk}(\sembrack{t_0},\sembrack{t_1})\\
\sembrack{\texttt{"}\ldots\texttt{"}}&{}={}&\mbox{``\ldots''}\\
\sembrack{\texttt{(enc~}t_0~\ldots~t_{n-1}~t_n\texttt{)}}&{}={}&
\enc{\sembrack{\texttt{(cat~}t_0~\ldots~t_{n-1}\texttt{)}}}{\sembrack{t_n}}\\
\sembrack{\texttt{(cat~}t\texttt{)}}&{}={}&\sembrack{t}\\
\sembrack{\texttt{(cat~}t_0~t_1~\ldots\texttt{)}}&{}={}&
(\sembrack{t_0},\sembrack{\texttt{(cat~}t_1~\ldots\texttt{)}})

\end{array}$$
\caption{S-expression Terms}\label{tab:trans}
\end{table}

The translation of S-expression terms is given in
Table~\ref{tab:trans}.  Figure~\ref{fig:ns responder} on
Page~\pageref{fig:ns responder} contains examples of \textsc{bca}
message terms.  Also see \textsc{term} in Table~\ref{tab:syntax},
Appendix~\ref{cha:bca syntax reference}.

The code that implements the Basic Crypto Algebra does not directly
implement an order-sorted algebra.  Instead, it implements a
many-sorted algebra and exports an order-sort algebra based on the
implementation.  Appendix~\ref{cha:bca} provides the complete details
of the implementation.

\section{Algebra Interface}

The details of each implementation of a {\cpsa} message algebra are
hidden by an interface.  This section presents the view of a term
algebra exposed by the interface.  Some aspects of the interface are
omitted from this discussion.  For example, each implementation of an
algebra must provide a means to read a term from an S-expression, and
write a term as an S-expression.  Also omitted are functions in the
interface that are specializations of a more general function added to
enhance performance.

Each algebra provides a predicate to determine if a term is a
variable, and another to determine if a term is an atom.  A fresh
variable generator is in the interface.  Given a generator state and a
term, it produces a clone of the term in which the variables have been
replaced with freshly generated ones.  It also returns the new
generator state.

\subsection{Equations}\label{sec:equations}

An algebra reports answers to unification and matching problems by
returning an order-sorted substitution or an error indicator.  A
different data structure is used for each problem, in this document
indicated by using~$\sigma$ for answers to the unification problem of
$\sigma(t_0)\equiv\sigma(t_1)$, and using~$\sigma_E$ for answers to the
matching problem of $\sigma_E(t_0)\equiv t_1$.  As the typical case is for
sets of equations to be solved, the unification and match functions
have been designed to allow an incremental approach to solving the
members of the set, by extending a substitution for one pair of
equated terms.  They have the following signatures:
$$\begin{array}{l}
\fn{unify}\colon
\mathcal{T}_\top(X)\times\mathcal{T}_\top(X)\times
(X\rightarrow\mathcal{T}_\top(X))\rightarrow
(X\rightarrow\mathcal{T}_\top(X))^?\\
\fn{match}\colon
\mathcal{T}_\top(X)\times\mathcal{T}_\top(Y)\times
(X\rightarrow\mathcal{T}_\top(Y))\rightarrow
(X\rightarrow\mathcal{T}_\top(Y))^?
\end{array}$$

For type~$T$, we write $T^?$ to describe the lifted type for~$T$ that
includes an error indicator.  In this formalism, sequences of length
less than two are used for the lifted type, and the error indicator is
the empty sequence.  In Haskell, the lifted type is
\index{maybe type}\texttt{Maybe~$T$}.

An answer to the matching problem is called an
\index{environment}\emph{environment}.  An environment differs from a
substitution produced as an answer to a unification problem in that it
may explicitly specify identity mappings, thus forbidding extensions
to the environment that conflicts with these mappings.  This
distinction is crucial for correctly answering matching problems by
iteratively extending an environment.

To support checks to see if terms are isomorphic via the match
function, the algebra interface includes the \emph{match variable
  renaming} predicate that tests an environment to see if it is a
one-to-one variable-to-variable order sorted substitution.

To support pruning, there is a function that given an environment and
a term, determines if there are variables in the term that are in the
domain of the environment.

\subsection{Term Internals}\label{sec:term internals}

The interface includes a function that returns the set of variables in
a term, and a function that returns the terms carried by a term.
Other subterms are accessed via position oriented
functions.  Recall that a position is a finite sequence of natural
numbers, and the message in~$t$ that occurs at~$p$, is
written~$t\termat p$.  The interface includes a data type for a
position that hides its implementation.  The interface also includes
the ancestors function $\fn{anc}(t,p)$ and the carried positions
function $\fn{carpos}(t,t')$ as defined in the specification.

Each algebra provides a way to obtain a set of positions at which a
subterm occurs within a term, and a way to replace the subterm at a
given position with another term.  These functions are used to
generalize by variable separation.

\begin{defn}[All Positions]\index{all positions}
Given a term~$t$, the set of positions at which~$t$ occurs in~$t'$ is
$\fn{allpos}(t,t')$, where
$$\fn{allpos}(t,t')=\left\{
\begin{array}{ll}
\{\seq{}\}&\mbox{if $t'\equiv t$, else}\\
\multicolumn{2}{l}{\{\seq{i}\append p \mid
p\in\fn{allpos}(t,t_i),i<n\}} \\
& \mbox{if $t'=f(t_0,\ldots,t_{n-1})$, else}\\
\{\}&\mbox{otherwise.}
\end{array}\right.$$
\end{defn}

\begin{defn}[Replace]\index{replace}
Given terms~$t$ and~$t'$, and position~$p$, the term that results from
replacing the term at~$p$ with~$t$ in~$t'$, is
$\fn{replace}(t,p,t')$, where
$$\begin{array}{l}
\fn{replace}(t,\seq{},t')=t;\\
\fn{replace}(t,\seq{i}\append p,f(t_0,\ldots,t_{n-1}))=
f(t'_0,\ldots,t'_{n-1})\mbox{ where}\\
t'_j=\left\{
\begin{array}{ll}
\fn{replace}(t,p,t_i)&\mbox{if $i=j$;}\\
t_j&\mbox{otherwise.}
\end{array}\right.
\end{array}$$
\end{defn}

\subsection{Encryptions and Derivations}\label{sec:encryptions}

Finally, the remaining functions in the interface are the ones that
expose the encryption oriented properties of terms.  The
\emph{decryption key} function returns the key used to decrypt a term
if it is an encryption, otherwise it returns an error indicator.  The
\emph{encryptions} function returns the set of encryption terms
carried by a term, each one paired with its encryption key.  The
penetrator derivable function from the section in the specification of
the same name is in the interface.  Given a derivable predicate that
has been specialized with a given set of supported terms and a set of
atoms to avoid, a target term, and a source term, the
\emph{protectors} function returns an error indicator if the target is
carried by the source outside of an encryption, where the derivable
predicate is used to determine if a decryption key can be used to
expose the target.  Otherwise, it returns a set of encryptions in the
source that carry the target and have underivable decryption keys.  If
two encryptions protect the target, only the outside one is returned.
The inside encryption is the one that is carried by the outside
encryption.  Pseudo code for the decryption key and the protectors
functions is in the specification.

\chapter{Protocols and Preskeletons}\label{cha:prots and preskels}

Terms over an order-sorted signature extended from a message signature
describe key data structures in the {\cpsa} program.  Given a message
signature that defines the sort \dom{mesg} and the atoms, the
additional sorts and operations are in the {\cpsa} Signature in
Table~\ref{tab:strands}.  The signature uses the sort $s~\dom{list}$
for sequences of terms of sort~$s$, and the sort $s~\dom{set}$ for
injective sequences of terms of sort~$s$.  The sorts used in the
signature are in Table~\ref{tab:sorts}.

\begin{table}
\begin{center}
Additional sort symbols: \dom{evt}, \dom{role},
\dom{maplet}, \dom{instance}, \\
\dom{node}, \dom{ordering}, and \dom{preskel} \\[1ex]
$\begin{array}{rl}
\outbnd\colon\dom{mesg}\rightarrow\dom{evt}\hspace{2\arraycolsep}
\inbnd\colon\dom{mesg}\rightarrow\dom{evt}&
\cn{r}\colon\dom{evt}\dom{list}\times\dom{atom}\dom{set}\times
\dom{atom}\dom{set}\rightarrow\dom{role}\\
\cn{m}\colon\dom{mesg}\times\dom{mesg}\rightarrow\dom{maplet}&
\cn{i}\colon\dom{role}\times\dom{nat}\times\dom{maplet}\dom{set}
\rightarrow\dom{instance}\\
\cn{n}\colon\dom{nat}\times\dom{nat}\rightarrow\dom{node}
&\cn{o}\colon\dom{node}\times\dom{node}\rightarrow\dom{ordering}
\end{array}$\\
$\cn{k}\colon\dom{role}\dom{set}\times
\dom{instance}\dom{list}\times
\dom{ordering}\dom{set}\times\dom{atom}\dom{set}\times
\dom{atom}\dom{set}\rightarrow\dom{preskel}$\\[1ex]
\end{center}
\caption{{\cpsa} Signature}\label{tab:strands}
\end{table}

Some of the terms over a {\cpsa} signature are not
\index{well-formed}well-formed, and omitted from interpretation.  The
text describing a term of a sort includes the conditions for it being
well-formed.

In what follows, the external syntax for protocols is presented, and
later, its translation into terms over a {\cpsa} signature.  For
preskeletons, the internal representation is presented first, followed
by its external syntax.

\begin{table}
\begin{center}
\begin{tabular}{rl}
\dom{mesg}& the sort of all messages (implementation of $\top$)\\
\dom{evt}& a transmission or reception event\\
\dom{trace}& a sequence of events used in a role\\
\dom{role}& a trace, a non-originating set, and a uniquely-originating
set\\
\dom{protocol}& a set of roles\\
\dom{nat}& a natural number\\
\dom{maplet}& a map from a role variable to a preskeleton term\\
\dom{instance}& a strand's trace and inheritance as instantiated from a role\\
\dom{node}& a pair of numbers, a strand identifier and a strand position\\
\dom{ordering}&a causal ordering between a pair of nodes\\
\dom{preskel}& a preskeleton
\end{tabular}
\caption{Protocol and Preskeleton Sorts}\label{tab:sorts}
\end{center}
\end{table}

\section{Protocols}

A protocol defines the patterns of allowed behavior for
non-adversarial participants, called the \emph{regular} participants.
The behavior of each regular participant is an instance of a protocol
template, called a role.  Figure~\ref{fig:ns roles} displays the roles
that make up the Needham-Schroeder protocol.

\begin{figure}
\begin{center}
\includegraphics{cpsadiagrams-0.mps}\hfil
\includegraphics{cpsadiagrams-1.mps}
\caption{Needham-Schroeder Initiator and Responder Roles}
\label{fig:ns roles}
\end{center}
\end{figure}

In S-expression syntax, a protocol is a named set of roles and is
defined by the \texttt{defprotocol} form.  See \textsc{protocol} in
Table~\ref{tab:syntax}, Appendix~\ref{cha:bca syntax reference}.
\begin{center}
\begin{tabular}{l}
\verb|(defprotocol ns basic|\\
\verb|  (defrole init| \ldots\texttt{)}\\
\verb|  (defrole resp| \ldots\texttt{))}
\end{tabular}
\end{center}

The name of this protocol (\textsc{id}) is \texttt{ns}, and the second
identifier (\textsc{alg}) names the message algebra in use.  The
identifier for the Basic Crypto Algebra is \texttt{basic}.

During the reading process, the appropriate algebra is implicitly
bound to the internal representation of a protocol and many data
structures derived from it.  The protocol name is used at read time to
bind it with its usages, and for output and error messages, but is
otherwise unused and thus omitted from the design specification. The
internal representation of a \index{protocol}protocol is simply a set
of roles---as a term of sort \dom{role~set} in Table~\ref{tab:strands}.

The S-expression syntax for a role has a name, a declared set of
variables, and a trace that provides a template for the behavior of
its instances.  A trace is a non-empty sequence of
events\index{event}, either a message transmission or a reception.  An
outbound\index{outbound} term is \texttt{(send $t$)} and an
inbound\index{inbound} message with term~$t$ is \texttt{(recv $t$)}.
The translations of events are $\outbnd\sembrack{t}$ and
$\inbnd\sembrack{t}$ respectively, where~$\sembrack{t}$ is the
translation of the S-expression~$t$ into an term of sort~\dom{mesg} in
Table~\ref{tab:strands}.  Needham-Schroeder responder's role in
S-expression syntax is in Figure~\ref{fig:ns responder}.

\begin{figure}
\begin{quote}
\begin{verbatim}
(defrole resp (vars (b a name) (n2 n1 text))
  (trace (recv (enc n1 a (pubk b)))
         (send (enc n1 n2 (pubk a)))
         (recv (enc n2 (pubk b)))))
\end{verbatim}
\end{quote}
\caption{Needham-Schroeder Responder Role}
\label{fig:ns responder}
\end{figure}

Some atoms in a role have special properties.  The atoms
listed in the \texttt{non-orig} form are assumed to be
non-originating, and those in the \texttt{uniq-orig} form are assumed
to be uniquely originating.  The implications of these assumption is
as in the specification.

Internally, \index{role}role~$\cn{r}(C,N,U)$ has a trace~$C$, and two
sets of atoms,~$N$ and~$U$.  The atoms in~$N$ are assumed to be
non-originating, and the atoms in~$U$ are assumed to be uniquely
originating.  As with protocols, the name is used during input and
output, but omitted from this specification.

A role is \index{well-formed role}well-formed if it satisfies the
conditions listed for a role in the specification.  A protocol is
well-formed if no variable occurs in more than one role.  The external
syntax used by {\cpsa} uses variable renaming to create the illusion
that the same variable may occur in two roles.  In the external
syntax, two roles may share the same identifier.

Associated with each protocol is an implicit set of roles.  For each
term~$t$ in its algebra, there is a \label{def:listener
  role}\index{listener role}\index{role!listener}\emph{listener role}
of the form~$\fn{lsn}(t) = \cn{r}(\seq{\inbnd t,\outbnd t}, \seq{},
\seq{})$.  A listener role is used to assert that a term is not a
secret.  In the implementation, the only difference between a listener
role and non-listener roles is its name is the empty string, a fact
used when printing.

\section{Preskeletons}

The other key {\cpsa} data structure is a preskeleton---see the~\cn{k}
operator in Table~\ref{tab:strands}.  A preskeleton is used to encode
classes of protocol executions, including its shapes, the answers
produced by {\cpsa}.  One component of a preskeleton is its protocol,
and one component is a set of strands.  There are more components, but
the set's representation is presented next.

As in the specification, a sequence of instances represents a set of
strands.  The instance\index{instance} $\cn{i}(r,h,E)$ contains a
role~$r$, a positive number~$h$ called its height, the length of the
trace associated with the instance, and an
environment\index{environment}~$E$, a term of sort \dom{maplet}
\dom{set}.

The environment~$E$ is well-formed if it represents the order-sorted
substitution~$\sigma_E$ such that for every maplet $\cn{m}(x,y)$,
$\sigma_E x=y$.  Note that~$x$ is always a variable, unlike its analog
in the external syntax.  An instance is well-formed if its role is
well-formed, its environment is well-formed, and its height is not
greater that the length of its role's trace.

The set of strands in a preskeleton is represented by a sequence of
instances.  The identity of a strand\index{strand} is its position in
the sequence, which is where the description of its trace is located.
The \index{node}node $\cn{n}(s,p)$ is associated with the event
at position~$p$ in strand~$s$'s trace.  In other words, if~$I$ is
a sequence of instances, the event at $\cn{n}(s,p)$,
\label{def:evt}written~$\fn{evt}(I,\cn{n}(s,p))$, is $\sigma_E(C(p))$, where
$I(s)=\cn{i}(\cn{r}(C,N,U),h,E)$ and~$p<h\leq|C|$.  A node
associated with an inbound term is a \index{reception
  node}\emph{reception node}, and a node associated with an outbound
term is a \index{transmission node}\emph{transmission node}.  The term
stripped of its direction is written~$\fn{msg}(I,\cn{n}(s,p))$.  The
set of nodes in sequence~$I$ is $\{\cn{n}(s,p) \mid s<|I|,
I(s)=\cn{i}(r,h,E), p < h\}$.

The preskeleton\index{preskeleton}~$\cn{k}(P,I,O,N,U)$ contains a
protocol~$P$, a non-empty sequence of instances~$I$, a set of
communication orderings~$O$, a set of non-originating
terms\index{non-originating term}~$N$, and a set of uniquely
originating terms~$U$.  The node
ordering\index{ordering}\index{communication ordering} $o(n_0, n_1)$
asserts that~$n_0$ precedes~$n_1$, that the event at~$n_0$ is
outbound, the event at~$n_1$ is inbound, and~$n_0$ and~$n_1$
are on different strands.  The atoms in~$N$ are assumed to be
non-originating, and the atoms in~$U$ are assumed to be uniquely
originating.

Members of the set of communication orderings~$O$ relate nodes in
differing strands.  There is an implied ordering of nodes within the
same strand.  Strand succession orderings of the form
$\cn{o}(\cn{n}(s, p-1), \cn{n}(s, p))$, where $0 < p < h$ and~$h$ is
the height of strand~$s$ are implicit, and must not be in~$O$.
\index{strand succession orderings}

\index{graph!preskeleton}\index{preskeleton graph}
Associated with each preskeleton~$k$ is a graph.  The vertices of the
graph are the nodes of the instance sequence~$I$, and the edges are
the reverse of the both communication ordering~$O$ and the implied
strand succession orderings.  The edges are reversed because events in
a node's past are of interest when analyzing a node.  When the graph
is acyclic, the transitive asymmetric relation~$\kprec{k}$ of~$k$ is the
transitive closure of the graph, and $n_0\kprec{k} n_1$ asserts that the
message event at~$n_0$ precedes the one at~$n_1$.  (A preskeleton with
a graph that contains cycles is not well-formed.)

To be well-formed\index{well-formed preskeleton}, in addition to the
requirements on communication orderings listed above, a preskeleton
must satisfy the same conditions listed for a preskeleton in the
specification.

\begin{figure}
\begin{center}
\includegraphics{cpsadiagrams-4.mps}
\caption{Needham-Schroeder Shape ($K^{-1}_A$ uncompromised, $N_2$ fresh)}
\label{fig:ns shape}
\end{center}
\end{figure}

\begin{figure}
\begin{center}
With $A,A',A'',B,B',B'',C\colon
\dom{name},N_1,N_1',N_1'',N_2,N_2',N_2''\colon\dom{text}$:
\end{center}
$$
\begin{array}{r@{}c@{}l}
\fn{resp}&{}={}&\cn{r}(\fn{resp}_t, \seq{},\seq{})
\mbox{ where}\\
\fn{resp}_t&{}={}&\seq{
\inbnd\enc{N_1,A}{K_B},
\outbnd\enc{N_1,N_2}{K_A},
\inbnd\enc{N_2}{K_B}}
\end{array}
$$

$$
\begin{array}{r@{}c@{}l}
\fn{init}&{}={}&\cn{r}(\fn{init}_t, \seq{},\seq{})
\mbox{ where}\\
\fn{init}_t&{}={}&\seq{
\outbnd\enc{N_1',A'}{K_{B'}},
\inbnd\enc{N_1',N_2'}{K_{A'}},
\outbnd\enc{N_2'}{K_{B'}}}
\end{array}
$$

$$\begin{array}{r@{}c@{}l}
\multicolumn{3}{l}{\cn{k}(\seq{\fn{resp},\fn{init}},
I, O,\seq{K_{A''}^{-1}},\seq{N_2''})\mbox{ where}}\\
I&{}={}&\langle\cn{i}(\fn{resp}, 3, E),
\cn{i}(\fn{init}, 3, E')\rangle\\
E&{}={}&\seq{\cn{m}(A,A''),\cn{m}(B,B''),
\cn{m}(N_1,N_1''), \cn{m}(N_2,N_2'')}\\
E'&{}={}&\seq{\cn{m}(A',A''),\cn{m}(B',C),
\cn{m}(N_1',N_1''), \cn{m}(N_2',N_2'')}\\
O&{}={}&\seq{\cn{o}(\cn{n}(0,1),\cn{n}(1,1)),
\cn{o}(\cn{n}(1,2),\cn{n}(0,2))}
\end{array}$$
\caption{Needham-Schroeder Preskeleton}\label{fig:ns}
\end{figure}

A Needham-Schroeder shape in traditional Strand Space notation is in
Figure~\ref{fig:ns shape}, and its representation using order-sorted
terms is given in Figure~\ref{fig:ns}.

\subsection{Preskeleton S-Expression Syntax}

The \texttt{defskeleton} form in Table~\ref{tab:syntax},
Appendix~\ref{cha:bca syntax reference} is used to specify a
preskeleton in S-expression syntax.  (With the exception of the
initial problem statement, a preskeleton is always a skeleton.)
On output, a preskeleton Referring to
Table~\ref{tab:syntax}, the \textsc{id} in the preskeleton form names
a protocol.  It refers to the most recent protocol definition of that
name which precedes the preskeleton form in the input.  The
\textsc{id} in the \texttt{defstrand} form names a role.  The integer
in the strand form gives the height of the strand.  The sequence of
pairs of terms in the strand form specify an environment used to
construct the events in a strand from its role's trace.  The
first term is interpreted using the role's variables and the second
term uses the preskeleton's variables.  The environment used to
produce the strand's trace is derived by matching the second term
using the first term as a pattern.  The \texttt{deflistener} form
creates an instance of a listener role for the given term.

The \texttt{precedes} form specifies members of the node relation.
The first integer in a node identifies the strand using the order in
which strands are defined in the \texttt{defskeleton} form.

\begin{figure}
\begin{quote}
\begin{verbatim}
(defskeleton ns (vars (n2 n1 text) (a b b-0 name))
  (defstrand resp 3 (n2 n2) (n1 n1) (b b) (a a))
  (defstrand init 3 (n1 n1) (n2 n2) (a a) (b b-0))
  (precedes ((0 1) (1 1)) ((1 2) (0 2)))
  (non-orig (privk a))
  (uniq-orig n2))
\end{verbatim}
\end{quote}
\caption{Needham-Schroeder \texttt{defskeleton}}\label{fig:defns}
\end{figure}

A variable may occur in more then one role within a protocol.  The
reader performs a renaming so as to ensure these occurrences do not
overlap.  Furthermore, the maplets used to specify a strand need not
specify how to map every role variable.  The reader inserts missing
mappings, and renames every preskeleton variable that also occurs in a
role of its protocol.  The sort of every preskeleton variable that
occurs in the \texttt{non-orig} or \texttt{uniq-orig} list or in a
maplet must be declared, using the \texttt{vars} form.

Needham-Schroeder shape in S-expression syntax is displayed in
Figure~\ref{fig:defns}.  The effect of reader renaming is shown in
Figure~\ref{fig:ns} by adding primes to variables.

The \textsc{prot-alist}, \textsc{role-alist}, and \textsc{skel-alist}
productions in Table~\ref{tab:syntax} are Lisp style association lists,
that is, lists of key-value pairs, where every key is a symbol.
Key-value pairs with unrecognized keys are ignored, and are available
for use by other tools.  On output, unrecognized key-value pairs are
preserved when printing protocols, but elided when printing
preskeletons.

See the {\cpsa} Primer for more examples of {\cpsa} external syntax.

\chapter{Reductions}\label{cha:reductions}

This chapter describes the implementation-oriented refinements made to
support reduction that are considered too detailed to be included in
the specification.

In the {\cpsa} implementation and the design formalism, every
preskeleton includes a link to its protocol.  Two preskeletons are not
related by a homomorphism unless they specify the same protocol.

For each preskeleton~$k$, the implementation maintains an
\index{origination map} \emph{origination map}, $\orig(k,t)$.  It maps
each of the preskeleton's uniquely originating terms to the set of
nodes at which it originates.  For hulled preskeletons, the range of
this map must contain singleton sets or the empty set.  The
origination map returns an error indicator when given a term not assumed
to be uniquely originating, a feature used to check the
implementation's consistency.

Each preskeleton contains the state of a variable generator.  It's
used to ensure a source of fresh variables for any preskeleton derived
from it.

\section{Preskeleton Reductions}\label{sec:preskeleton reductions}

Given a well-formed preskeleton, an attempt is made to convert it into
a set of skeletons.  This section describes a few implementation
details omitted from the specification.

The implementation uses sequences to represent some sets.  The
function \index{nub}\fn{nub} removes duplicates from a sequence.

\subsection{Substitution}

The function~$\ops{S}_\sigma$ applies the order-sorted substitution~$\sigma$
to a preskeleton.
$$
\begin{array}{r@{}c@{}l}
\ops{S}_\sigma(\cn{k}(P, I, O, N, U))&{}={}&
\cn{k}(P, \ops{S}_\sigma\circ I, O, \fn{nub}(\sigma\circ N),
\fn{nub}(\sigma\circ U))\\
\ops{S}_\sigma(\cn{i}(r,h,E))&{}={}&\cn{i}(r, h, \ops{S}_\sigma\circ E)\\
\ops{S}_\sigma(\cn{m}(x,y))&{}={}&\cn{m}(x,\sigma(y))
\end{array}
$$

The substitution is permitted as long as it preserves the nodes at
which each uniquely originating term originates.  In other words, the
substitution is permitted only if for each uniquely originating
term~$t$ in~$k$, $\orig(k,t)\subseteq\orig(k', \sigma(t))$, where
$k'=\ops{S}_\sigma(k)$.  The implicit homomorphism is
$(\idphi,\sigma)$, where~$\idphi$ is the identity strand map for~$k$.

\subsection{Compression}

The function~$\ops{C}_{s,s'}$ compresses~$s$ into~$s'$ in a preskeleton.
$$
\begin{array}{r@{}c@{}l}
\ops{C}_{s,s'}(\cn{k}(P, I, O, N, U))&{}={}&
\cn{k}(P, I\circ\phi'_s, \ops{C}_{\phi}\circ O, N, U)\\
\ops{C}_\phi(\cn{o}(n_0, n_1))&{}={}
&\cn{o}(\ops{C}_\phi(n_0), \ops{C}_\phi(n_1))\\
\ops{C}_\phi(\cn{n}(s, p))&{}={}&\cn{n}(\phi(s), p)\\
\phi(j)=\phi_{s,s'}(j)&{}={}&\left\{
\begin{array}{ll}
\phi_s(s')&\mbox{if $j=s$}\\
\phi_s(j)&\mbox{otherwise}
\end{array}\right.\\
\phi_s(j)&{}={}&\left\{
\begin{array}{ll}
j-1&\mbox{if $j>s$}\\
j&\mbox{otherwise}
\end{array}\right.\\
\phi'_s(j)&{}={}&\left\{
\begin{array}{ll}
j+1&\mbox{if $j\geq s$}\\
j&\mbox{otherwise}
\end{array}\right.
\end{array}
$$ where the trace of~$I(s)$ is a prefix of the trace of~$I(s')$.
Although not shown, orderings of the form
$\cn{o}(\cn{n}(s,p),\cn{n}(s,p'))$ are removed from the ordering when
$p<p'$, so they do not cause the output preskeleton to fail to be
well-formed.  The implicit homomorphism is
$(\phi_{s,s'},\idsigma)$, where~$\idsigma$ is the identity
substitution.  Note that $\phi_{s,s'}\circ\phi'_s=\idsigma$.

\subsection{Transitive Reduction}

The function~$\ops{R}$ performs a transitive reduction on a preskeleton's
node relation.  The transitive reduction\index{transitive reduction}
of an ordering is the minimal ordering such that both orderings have
the same transitive closure.  Here, communication orderings implied by
transitive closure are removed.

The transitive reduction of a skeleton is isomorphic to the skeleton.
The reduction is performed to speed up the code that checks for
isomorphisms.  When two skeletons are transitively reduced, and
isomorphic, they have the same number of communication orderings.

In the implementation, transitive reduction is the last operation
applied to a preskeleton during the process of converting a
preskeleton into a pruned skeleton.  Isomorphism testing is only
performed on pruned skeletons.

\section{Augmentation}

The function~$\ops{A}_{i,n}$ augments a preskeleton with a new strand.
It appends the instance~$i$ to the sequence of instances, adds a node
ordering, and adds atoms as specified by the role.  The function
orders the last node in the strand before some node in the
preskeleton.

$$
\begin{array}{r@{}c@{}l}
\ops{A}_{i,n}(\cn{k}(P, I, O, N, U))&{}={}&
\cn{k}(P, I\append\langle i\rangle,
\langle \cn{o}(\cn{n}(|I|,h-1),n)
\rangle \append O,N_1, U_1)\\
i&{}={}&\cn{i}(\cn{r}(C,N_0,U_0),h,E)\\
N_1&{}={}&\fn{nub}(N\append\sigma_e\circ N_0)\\
U_1&{}={}&\fn{nub}(U\append\sigma_e\circ U_0)
\end{array}
$$ Although not shown, elements in~$N_1$ that contain a variable that
does not occur in some event in the constructed preskeleton
are dropped, as are elements in~$U_1$ that are not carried in some
event in the constructed preskeleton.  As with the preskeleton
reductions in Section~\ref{sec:preskeleton reductions}, it is
straightforward to derive the homomorphism associated with a
successful augmentation.

\section{Generalization}\label{sec:generalization}

The generalization process is the only part of the algorithm that
requires non-isomorphic homomorphisms to be explicitly represented.
To make this possible, every preskeleton contains two additional
fields not yet described, a link to the point-of-view skeleton, and a
{\pov} strand map.  The {\pov} strand map is the second component of
the homomorphism from the point-of-view skeleton to the preskeleton.
The first component can be generated via matching using the
point-of-view skeleton.

The {\pov} strand map is maintained by every reduction applied to a
preskeleton.  The compression reduction is the only one that
requires careful thought.

\chapter{Search Strategy}\label{cha:search}

The top-level loop maintains two sequences of skeletons, a to do list,
and a set of skeletons that have already been seen.  When the to do
list is empty, the loop exits.

For each problem statement, {\cpsa} attempts the convert the
preskeleton into a skeleton.  If the conversion fails, an error is
signaled.  Otherwise, a search is started with the skeleton as the
{\pov} skeleton of Section~\ref{sec:generalization}.  The top-level loop
starts with the {\pov} skeleton as the single member of the to do
list and the seen set.

The following steps constitute one iteration of the top-level loop.
The first skeleton on the to do list is removed and is the subject of
the iteration.  If the skeleton is unrealized, contraction and
augmentation are used to compute its cohort.  Otherwise, generalization
reductions are tried in an effort to make one generalization step.
The generated skeletons are the subject's children.  Each child that
is isomorphic to a member of the seen set is dropped.  The other
children are added to the seen set and to the end of the to do list.
The final step in the iteration is to print the subject skeleton.

When the subject in unrealized, the test node is selected as follows.
The strands are considered in reverse order, and the first unrealized
node in one of the strands is used as the test node.  The program
happens to try atoms for critical messages before it tries
encryptions, but the overall results should be independent of the
algorithms used to select test nodes and critical messages, an assertion
that requires justification.  The encryptions considered as critical
messages are obtained using the \emph{encryptions} function in the
algebra interface.

\chapter{Visualization}\label{cha:viz}

This section describes the Causally Intuitive Preskeleton Layout
algorithm used to generate visualizations of preskeletons.  The
algorithm is simple to implement and explain, and because it is
designed for preskeletons, it produces better results than is
available from generic graph layout algorithms, such as the ones used
by Graphviz~\cite{GansnerNorth00}.

The preskeleton is prepared by performing a transitive reduction on
its ordering relation.  Communication edges implied by transitive
closure are removed.  The result is called a Hasse diagram.

The Hasse diagram is created by considering each communication edge.
If there is a path from its source to its destination that does not
traverse the edge, the edge is deleted.

To simplify the task, each strand is laid out vertically with early
nodes above later ones.  The strands are horizontally placed in the
same order as they appear in the preskeleton.  The spacing between
successive nodes on a strand is the same, as is the horizontal spacing
between strands.  The vertical position of a node is called its rank.

Within this framework, the simplest layout algorithm is to use the
position of the node in its strand as its rank.  When using this
layout strategy, the only difficultly occurs when a node ordering
arrow crosses over a node in an unrelated strand.  To avoid ambiguity,
arrows that cross strands are curved.

When using position based ranking, the result often contains upward
sloping arrows.  Within a strand, no node that is after a node is
above the node, but with upward sloping arrows, this property no
longer holds.  The motivation for the Causally Intuitive Preskeleton
Layout algorithm is that eliminating upward sloping arrows makes
causal relations easier to grasp.

The layout algorithm has two phases.  The first phase stretches
strands so as to eliminate upwardly sloping arrows, and the second
phase compresses them so as to eliminate some unnecessary stretching.
Without phase two, some nodes early in a strand appear to be oddly
separated from others in the strand.

Each phase starts with a to do list containing every node in the
preskeleton.  For each node in the to do list, if conditions are met,
it updates the current node ranking and adds nodes to the to do list.
The phase is finished when the to do list is empty.

Phase I starts with the position based ranking~$r(s,p)=p$.  Let~$P(n)$
be the set the predecessors of node~$n$, excluding the nodes on the
strand of~$n$.  Let~$P_r(n)=\{r(n')\mid n'\in P(n)\}$ be the ranks
of~$P(n)$.  The stretch rule is considered for each element in the
to do list.

The stretch rule applies to node~$n_1$ if $r(n_1)<h$, where
$h=\max(\{r(n_1)\}\cup P_r(n_1))$.  In that case, the ranking is updated
so that $r(n_1)=h$, and the linearize rule is applied to the next strand
node if it exists.

The linearize rule applies to node~$n_1$ if $r(n_1)\leq r(n_0)$,
where~$n_0$ is the previous strand node.  In that case, the ranking is
updated so that $r(n_1)=r(n_0)+1$, the to do list is augmented with
elements in~$S(n_1)$, and the linearize rule is applied to the next
strand node if it exists, where~$S(n)$ is the set the successors of
node~$n$, excluding the nodes on the strand of~$n$.

In phase II, the compress rule is considered for each element in the
to do list.  It applies to node~$n_1$ with a next strand node of~$n_2$
if $r(n_1)<h$, where $h=\min(\{r(n_2)-1)\}\cup S_r(n_1))$ and~$S_r(n)$
is the ranks of~$S(n)$.  In that case, the ranking is updated so that
$r(n_1)=h$, and the to do list is augmented with elements in~$P(n_1)$
and the previous strand node of~$n_1$ if it exists.

\chapter*{Acknowledgement}

Carolyn Talcott provided valuable feedback on drafts of this document.

\appendix

\chapter{Basic Crypto Algebra Syntax Reference}\label{cha:bca syntax reference}

The complete syntax for the analyzer using the Basic Crypto Algebra is
shown in Table~\ref{tab:syntax}.  The start grammar symbol is
\textsc{file}, and the terminal grammar symbols are: \textsc{(, ),
  symbol, string, integer,} and the constants set in typewriter font.

The \textsc{prot-alist}, \textsc{role-alist}, and \textsc{skel-alist}
productions are Lisp style association lists, that is, lists of
key-value pairs, where every key is a symbol.  Key-value pairs with
unrecognized keys are ignored, and are available for use by other
tools.  On output, unrecognized key-value pairs are preserved when
printing protocols, but elided when printing skeletons.

\begin{table}
\newcommand{\sym}[1]{\textup{\texttt{#1}}}
\begin{center}\scshape
\begin{tabular}{rcl}
file&$\leftarrow$&form+
\\form&$\leftarrow$&
$\mbox{comment}\mid\mbox{protocol}\mid\mbox{skeleton}$
\\ comment&$\leftarrow$&
(\sym{comment}~\ldots)
\\ protocol&$\leftarrow$&
(\sym{defprotocol} id alg role+ prot-alist)
\\ id&$\leftarrow$&symbol
\\ alg&$\leftarrow$&symbol
\\ role&$\leftarrow$&
(\sym{defrole} id vars trace role-alist)
\\ vars&$\leftarrow$&
(\sym{vars} decl$\ast$)
\\ decl&$\leftarrow$&
(id+ sort)
\\ sort&$\leftarrow$&
$\sym{text} \mid \sym{data} \mid \sym{name} \mid \sym{skey}
\mid \sym{akey}\mid\sym{mesg}$
\\ trace&$\leftarrow$&(\sym{trace} event+)
\\ event&$\leftarrow$&
$(\sym{send}\mbox{ term})\mid(\sym{recv}\mbox{ term})$
\\ term&$\leftarrow$&
$\mbox{id}\mid(\sym{pubk}\mbox{ id})
\mid(\sym{privk}\mbox{ id})
\mid(\sym{invk}\mbox{ id})$
\\ &$\mid$&$(\sym{ltk}\mbox{ id id})\mid\mbox{string}\mid(\sym{cat}\mbox{ term+})$
\\ &$\mid$&$(\sym{enc}\mbox{ term+ term})$
\\ role-alist&$\leftarrow$&$
(\sym{non-orig}\mbox{ ht-term}\ast)\mbox{ role-alist}$
\\ &$\mid$&$(\sym{uniq-orig}\mbox{ term}\ast)\mbox{ role-alist}\mid\ldots$
\\ ht-term&$\leftarrow$&term${}\mid{}$(integer term)
\\ prot-alist&$\leftarrow$&$\ldots$
\\ skeleton&$\leftarrow$&
(\sym{defskeleton} id vars
\\ &&\qquad strand+ skel-alist)
\\ strand&$\leftarrow$&
(\sym{defstrand} id integer maplet$\ast$)
\\ &$\mid$&(\sym{deflistener} term)
\\ maplet&$\leftarrow$&
(term term)
\\ skel-alist&$\leftarrow$&$(\sym{non-orig}\mbox{ term}\ast)\mbox{ skel-alist}$
\\ &$\mid$&$(\sym{uniq-orig}\mbox{ term}\ast)\mbox{ skel-alist}$
\\ &$\mid$&$(\sym{precedes}\mbox{ node-pair}\ast)\mbox{ skel-alist}\mid\ldots$
\\ node-pair&$\leftarrow$&
(node node)
\\ node&$\leftarrow$&
(integer integer)
\end{tabular}
\end{center}
\caption{{\cpsa} Syntax}\label{tab:syntax}
\end{table}

The contents of a file can be interpreted as a sequence of
S-expressions.  The S-expressions used are restricted so that most
dialects of Lisp can read them, and characters within symbols and
strings never need quoting.  Seven-bit \textsc{ascii} is used to
encode characters.  Every list is proper.  An S-expression atom is
either a \textsc{symbol}, an \textsc{integer}, or a \textsc{string}.
The characters that make up a symbol are the letters, the digits, and
the special characters in ``\verb|-*/<;=>;!?:$%_&~^+|''.  A symbol may
not begin with a digit or a sign followed by a digit.  The characters
that make up a string are the \textsc{ascii} printing characters
omitting double quote and backslash.  Double quotes delimit a string.
A comment\index{comments} begins with a semicolon, or is an
S-expression list at top-level that starts with the \texttt{comment}
symbol.


\chapter{The Basic Crypto Many-Sorted Algebra}\label{cha:bca}

The implementation uses a many-sorted algebra.  The many-sorted
message algebra described here is a reduction of the order-sorted
message algebra in Table~\ref{tab:bca} using the method described
in~\cite[Section~4]{GoguenMeseguer92}.  The order-sorted message
signature is reproduced in Table~\ref{tab:order-sorted} in a form that
uses prefix notation for every term formed using an operator.  In the
related many-sorted signature in Table~\ref{tab:many-sorted}, the
inclusion function\index{inclusion function} symbols are \cn{text},
\cn{data}, \cn{name}, \cn{skey}, and \cn{akey}.  Section~4 of the
paper describes the sense in which algebras that model the many-sorted
signature are essentially the same as the ones that model the
order-sorted message signature.

\begin{table}
\begin{center}
Base sort symbols: \dom{name}, \dom{text}, \dom{data}, \dom{skey},
\dom{akey}\\
Non-base sort symbol: \dom{mesg} \\[1ex]
Subsorts: \dom{name}, \dom{text}, \dom{data}, \dom{akey},
$\dom{skey}<\dom{mesg}$\\[1ex]
$\cn{pubk}\colon\dom{name}\rightarrow\dom{akey}\quad
\cn{invk}\colon\dom{akey}\rightarrow\dom{akey}\quad
\cn{ltk}\colon\dom{name}\times\dom{name}\rightarrow\dom{skey}$\\
$\cn{enc}\colon\dom{mesg}
\times\dom{mesg}\rightarrow\dom{mesg}\quad
\cn{cat}\colon\dom{mesg}
\times\dom{mesg}\rightarrow\dom{mesg}\quad C_i\colon\dom{mesg}$\\[1ex]
Axiom: $\cn{invk}(\cn{invk}(x))\approx x$ for $x\colon\dom{akey}$\\
\caption{Basic Crypto Order-Sorted Signature}\label{tab:order-sorted}
\end{center}
\end{table}

Terms are constructed from a set $I$ of identifiers\index{identifiers}
and a set of functions symbols.  The symbols of arity one are
\cn{text}, \cn{data}, \cn{name}, \cn{skey}, \cn{akey}, \cn{pubk}, and
\cn{invk}.  The symbols of arity two are \cn{ltk}, \cn{cat}, and
\cn{enc}.  The signature is given in Table~\ref{tab:many-sorted}.
Grammar rules define the terms used by this algebra.

\begin{table}
\begin{center}
Sort symbols: \dom{name}, \dom{text}, \dom{data},
\dom{skey}, \dom{akey}, and \dom{mesg}\\[1ex]
$\cn{pubk}\colon\dom{name}\rightarrow\dom{akey}\quad
\cn{invk}\colon\dom{akey}\rightarrow\dom{akey}\quad
\cn{ltk}\colon\dom{name}\times\dom{name}\rightarrow\dom{skey}$\\
$\cn{enc}\colon\dom{mesg}
\times\dom{mesg}\rightarrow\dom{mesg}\quad
\cn{cat}\colon\dom{mesg}
\times\dom{mesg}\rightarrow\dom{mesg}\quad C_i\colon\dom{mesg}$\\
$\cn{name}\colon\dom{name}\rightarrow\dom{mesg}\quad
\cn{text}\colon\dom{text}\rightarrow\dom{mesg}\quad
\cn{data}\colon\dom{data}\rightarrow\dom{mesg}$\\
$\cn{skey}\colon\dom{skey}\rightarrow\dom{mesg}\quad
\cn{akey}\colon\dom{skey}\rightarrow\dom{mesg}$\\[1ex]
Axiom: $\cn{invk}(\cn{invk}(x))\approx x$ for $x\colon\dom{akey}$\\
\caption{Basic Crypto Many-Sorted Signature}\label{tab:many-sorted}
\end{center}
\end{table}

The set of asymmetric keys $K$ is defined as follows.
$$ K \leftarrow I \mid \cn{pubk}(I) \mid \cn{invk}(I) \mid
\cn{invk}(\cn{pubk}(I)) $$ The key $\cn{invk}(x)$ is the inverse of
the asymmetric key~$x$, and $\cn{pubk}(x)$ is principal $x$'s public
key.

Each occurrence of an identifier in a term is associated with a sort
symbol.  The context in which an identifier occurs determines the
sort.  The sort symbols are \dom{text}, \dom{data}, \dom{name},
\dom{akey}, \dom{skey}, and \dom{mesg}, where a name refers to a
principal.  An identifier occurrence in an asymmetric key of the form
$\cn{pubk}(x)$ has sort \dom{name}, otherwise it has sort \dom{akey}.

The set of atoms $B$ is defined as follows.
$$ B \leftarrow \cn{text}(I) \mid \cn{data}(I) \mid \cn{name}(I) \mid
\cn{skey}(I)\mid \cn{skey}(\cn{ltk}(I, I)) \mid \cn{akey}(K) $$ The
atom $\cn{skey}(\cn{ltk}(x, y))$ is a symmetric, long term key
shared between two principals~$x$, and~$y$.  The occurrence of $x$ in
$\cn{text}(x)$ has sort \dom{text}, sort \dom{data} for
$\cn{data}(x)$, sort \dom{name} for $\cn{name}(x)$, and sort
\dom{skey} for $\cn{skey}(x)$.  The occurrences of $x$ and $y$ in
$\cn{skey}(\cn{ltk}(x,y))$ both have sort \dom{name}.

The set of terms $T$ is defined as follows.
$$ T \leftarrow I \mid B \mid Q \mid \cn{cat}(T, T) \mid \cn{enc}(T,
T)$$ where $Q$ is the set of tags, represented by quoted string
literals.  The second argument in \cn{enc} is a term for a key.  A
term of the form $x$ is called an indeterminate, and the identifier
occurrence has sort~\dom{mesg}.

The terms of interest are \index{well-formed term}well-formed.  A term
is \emph{well-formed} if every occurrence of each identifier has the
same sort.  An example of a non-well-formed term is
$\cn{cat}(x,\cn{akey}(x))$ because the identifier $x$ occurs with two
sorts, \dom{mesg} and \dom{akey}.  A pair of well-formed terms are
\emph{compatible}\index{compatible terms} if every identifier that
occurs in both terms occurs with the same sort.

A term is a variable if it specifies an identifier and its sort.
$$ V \leftarrow I \mid \cn{text}(I)  \mid \cn{data}(I) \mid \cn{name}(I) \mid
\cn{skey}(I)\mid \cn{akey}(I) $$ When a term is well-formed, the same
variable is associated every occurrence of an identifier in a term.

\begin{figure}
$$
\begin{array}{l}
\cn{unify}(\ell, t, t') = \cn{unify\_aux}(\ell,\cn{chase}(\ell, t),
\cn{chase}(\ell, t'))
\\
\\\cn{chase}(\ell, x)=
\\ \qquad\mbox{let }t=\cn{lookup}(x,\ell)\mbox{ in}
\\ \qquad \mbox{if }x = t\mbox{ then }x
\mbox{ else }\cn{chase}(\ell, t)
\\\cn{chase}(\ell, \cn{invk}(t))=\cn{chase\_invk}(\ell, t)\hfill(!)
\\\cn{chase}(\ell, t)=t
\\
\\\cn{chase\_invk}(\ell, x)=\hfill(!)
\\ \qquad\mbox{let }t=\cn{lookup}(x,\ell)\mbox{ in}\hfill(!)
\\ \qquad \mbox{if }x = t\mbox{ then }\cn{invk}(x)
\mbox{ else }\cn{chase\_invk}(\ell, t)\hfill(!)
\\\cn{chase\_invk}(\ell, \cn{invk}(t))=\cn{chase}(\ell, t)\hfill(!)
\\\cn{chase\_invk}(\ell, t)=\cn{invk}(t)\hfill(!)
\\
\\\cn{lookup}(x,\seq{})=x
\\\cn{lookup}(x,(y,t)\cons\ell)=
\mbox{if }x=y\mbox{ then }t\mbox{ else }
\cn{lookup}(x, \ell)
\\
\\\cn{unify\_aux}(\ell,x,x) =\ell
\\\cn{unify\_aux}(\ell,x,t) =\mbox{if }\cn{occurs}(x, t)\mbox{ then raise failure else }
(x,t)\cons\ell
\\\cn{unify\_aux}(\ell,t,x) =\cn{unify\_aux}(\ell,x,t)
\\\cn{unify\_aux}(\ell,\cn{invk}(x),\cn{pubk}(y)) =
\cn{unify\_aux}(\ell, x, \cn{invk}(\cn{pubk}(y)))\hfill(!)
\\\cn{unify\_aux}(\ell,\cn{pubk}(x),\cn{invk}(y)) =
\cn{unify\_aux}(\ell, y, \cn{invk}(\cn{pubk}(x)))\hfill(!)
\\\cn{unify\_aux}(\ell,f(t,\ldots),f(t',\ldots)) =
\cn{unify\_list}(\ell,\seq{t,\ldots},\seq{t',\ldots})
\\\cn{unify\_aux}(\ell,t,t') =\mbox{raise failure}
\\
\\\cn{unify\_list}(\ell,\seq{},\seq{}) =\ell
\\\cn{unify\_list}(\ell,t\cons u,t'\cons u') =
\cn{unify\_list}(\cn{unify}(\ell,t,t'), u, u')
\\\cn{unify\_list}(\ell,u,u') =\mbox{raise failure}
\end{array}
$$
\caption{Unifier}\label{fig:unifier}
\end{figure}

There are efficient ways of implementing unification for this algebra
because there are efficient ways for implementing for unification in
equational theories representable by a convergent term rewrite
system~\cite{Fay79}. As long as terms are compatible, substitutions
produced by unifiers map an identifier that occurs with a given sort to a
term of the same sort.  Depending on the sort symbol, a substitution is
limited to the following forms:
$$
\begin{array}{cllll}
\dom{mesg}&I\mapsto I&I\mapsto B
&I\mapsto\cn{cat}(T,T)&I\mapsto\cn{enc}(T,T)\ldots\\
\dom{akey}&I\mapsto I&I\mapsto\cn{pubk}(I)
&I\mapsto\cn{invk}(I)&I\mapsto\cn{invk}(\cn{pubk}(I))\\
\dom{skey}&I\mapsto I&I\mapsto\cn{ltk}(I, I)\\
\dom{text}&I\mapsto I\\
\dom{data}&I\mapsto I\\
\dom{name}&I\mapsto I
\end{array}
$$

The current implementation uses an algorithm for unification without
equations described by Laurence Paulson~\cite[Page~381]{Paulson91}
with modifications to the unification functions as shown in
Figure~\ref{fig:unifier}, where $e\cons\ell$ means
$\seq{e}\append\ell$.  The function \cn{unify} calls \cn{unify\_aux}
in the unmodified version.

\begin{figure}
$$
\begin{array}{l}
\cn{match}(\ell,x,t) =
\\\qquad\mbox{if }\lnot\cn{bound}(x, \ell)\mbox{ then }(x,t)\cons\ell
\\\qquad\mbox{else if } \cn{lookup}(x,\ell)= t\mbox{ then }\ell
\\\qquad\mbox{else raise failure}
\\\cn{match}(\ell,\cn{invk}(x),\cn{pubk}(y)) =
\cn{match}(\ell, x, \cn{invk}(\cn{pubk}(y)))\hfill(!)
\\\cn{match}(\ell,f(t,\ldots),f(t',\ldots)) =
\cn{match\_list}(\ell,\seq{t,\ldots},\seq{t',\ldots})
\\\cn{match}(\ell,t,t') =\mbox{raise failure}
\\
\\\cn{bound}(x,\seq{})=\cn{false}
\\\cn{bound}(x,(y,t)\cons\ell)=
x=y\mbox{ or }\cn{bound}(x, \ell)
\\
\\\cn{match\_list}(\ell,\seq{},\seq{}) =\ell
\\\cn{match\_list}(\ell,t\cons u,t'\cons u') =
\cn{match\_list}(\cn{match}(\ell,t,t'), u, u')
\\\cn{match\_list}(\ell,u,u') =\mbox{raise failure}
\end{array}
$$
\caption{Matcher}\label{fig:matcher}
\end{figure}

\bibliography{cpsa}
\bibliographystyle{plain}

\printindex

\end{document}
